\documentclass[]{article}
\usepackage{lmodern}
\usepackage{amssymb,amsmath}
\usepackage{ifxetex,ifluatex}
\usepackage{fixltx2e} % provides \textsubscript
\ifnum 0\ifxetex 1\fi\ifluatex 1\fi=0 % if pdftex
  \usepackage[T1]{fontenc}
  \usepackage[utf8]{inputenc}
\else % if luatex or xelatex
  \ifxetex
    \usepackage{mathspec}
  \else
    \usepackage{fontspec}
  \fi
  \defaultfontfeatures{Ligatures=TeX,Scale=MatchLowercase}
\fi
% use upquote if available, for straight quotes in verbatim environments
\IfFileExists{upquote.sty}{\usepackage{upquote}}{}
% use microtype if available
\IfFileExists{microtype.sty}{%
\usepackage{microtype}
\UseMicrotypeSet[protrusion]{basicmath} % disable protrusion for tt fonts
}{}
\usepackage[margin=1in]{geometry}
\usepackage{hyperref}
\hypersetup{unicode=true,
            pdftitle={House Price Precictions R Notebook},
            pdfborder={0 0 0},
            breaklinks=true}
\urlstyle{same}  % don't use monospace font for urls
\usepackage{color}
\usepackage{fancyvrb}
\newcommand{\VerbBar}{|}
\newcommand{\VERB}{\Verb[commandchars=\\\{\}]}
\DefineVerbatimEnvironment{Highlighting}{Verbatim}{commandchars=\\\{\}}
% Add ',fontsize=\small' for more characters per line
\usepackage{framed}
\definecolor{shadecolor}{RGB}{248,248,248}
\newenvironment{Shaded}{\begin{snugshade}}{\end{snugshade}}
\newcommand{\KeywordTok}[1]{\textcolor[rgb]{0.13,0.29,0.53}{\textbf{#1}}}
\newcommand{\DataTypeTok}[1]{\textcolor[rgb]{0.13,0.29,0.53}{#1}}
\newcommand{\DecValTok}[1]{\textcolor[rgb]{0.00,0.00,0.81}{#1}}
\newcommand{\BaseNTok}[1]{\textcolor[rgb]{0.00,0.00,0.81}{#1}}
\newcommand{\FloatTok}[1]{\textcolor[rgb]{0.00,0.00,0.81}{#1}}
\newcommand{\ConstantTok}[1]{\textcolor[rgb]{0.00,0.00,0.00}{#1}}
\newcommand{\CharTok}[1]{\textcolor[rgb]{0.31,0.60,0.02}{#1}}
\newcommand{\SpecialCharTok}[1]{\textcolor[rgb]{0.00,0.00,0.00}{#1}}
\newcommand{\StringTok}[1]{\textcolor[rgb]{0.31,0.60,0.02}{#1}}
\newcommand{\VerbatimStringTok}[1]{\textcolor[rgb]{0.31,0.60,0.02}{#1}}
\newcommand{\SpecialStringTok}[1]{\textcolor[rgb]{0.31,0.60,0.02}{#1}}
\newcommand{\ImportTok}[1]{#1}
\newcommand{\CommentTok}[1]{\textcolor[rgb]{0.56,0.35,0.01}{\textit{#1}}}
\newcommand{\DocumentationTok}[1]{\textcolor[rgb]{0.56,0.35,0.01}{\textbf{\textit{#1}}}}
\newcommand{\AnnotationTok}[1]{\textcolor[rgb]{0.56,0.35,0.01}{\textbf{\textit{#1}}}}
\newcommand{\CommentVarTok}[1]{\textcolor[rgb]{0.56,0.35,0.01}{\textbf{\textit{#1}}}}
\newcommand{\OtherTok}[1]{\textcolor[rgb]{0.56,0.35,0.01}{#1}}
\newcommand{\FunctionTok}[1]{\textcolor[rgb]{0.00,0.00,0.00}{#1}}
\newcommand{\VariableTok}[1]{\textcolor[rgb]{0.00,0.00,0.00}{#1}}
\newcommand{\ControlFlowTok}[1]{\textcolor[rgb]{0.13,0.29,0.53}{\textbf{#1}}}
\newcommand{\OperatorTok}[1]{\textcolor[rgb]{0.81,0.36,0.00}{\textbf{#1}}}
\newcommand{\BuiltInTok}[1]{#1}
\newcommand{\ExtensionTok}[1]{#1}
\newcommand{\PreprocessorTok}[1]{\textcolor[rgb]{0.56,0.35,0.01}{\textit{#1}}}
\newcommand{\AttributeTok}[1]{\textcolor[rgb]{0.77,0.63,0.00}{#1}}
\newcommand{\RegionMarkerTok}[1]{#1}
\newcommand{\InformationTok}[1]{\textcolor[rgb]{0.56,0.35,0.01}{\textbf{\textit{#1}}}}
\newcommand{\WarningTok}[1]{\textcolor[rgb]{0.56,0.35,0.01}{\textbf{\textit{#1}}}}
\newcommand{\AlertTok}[1]{\textcolor[rgb]{0.94,0.16,0.16}{#1}}
\newcommand{\ErrorTok}[1]{\textcolor[rgb]{0.64,0.00,0.00}{\textbf{#1}}}
\newcommand{\NormalTok}[1]{#1}
\usepackage{graphicx,grffile}
\makeatletter
\def\maxwidth{\ifdim\Gin@nat@width>\linewidth\linewidth\else\Gin@nat@width\fi}
\def\maxheight{\ifdim\Gin@nat@height>\textheight\textheight\else\Gin@nat@height\fi}
\makeatother
% Scale images if necessary, so that they will not overflow the page
% margins by default, and it is still possible to overwrite the defaults
% using explicit options in \includegraphics[width, height, ...]{}
\setkeys{Gin}{width=\maxwidth,height=\maxheight,keepaspectratio}
\IfFileExists{parskip.sty}{%
\usepackage{parskip}
}{% else
\setlength{\parindent}{0pt}
\setlength{\parskip}{6pt plus 2pt minus 1pt}
}
\setlength{\emergencystretch}{3em}  % prevent overfull lines
\providecommand{\tightlist}{%
  \setlength{\itemsep}{0pt}\setlength{\parskip}{0pt}}
\setcounter{secnumdepth}{0}
% Redefines (sub)paragraphs to behave more like sections
\ifx\paragraph\undefined\else
\let\oldparagraph\paragraph
\renewcommand{\paragraph}[1]{\oldparagraph{#1}\mbox{}}
\fi
\ifx\subparagraph\undefined\else
\let\oldsubparagraph\subparagraph
\renewcommand{\subparagraph}[1]{\oldsubparagraph{#1}\mbox{}}
\fi

%%% Use protect on footnotes to avoid problems with footnotes in titles
\let\rmarkdownfootnote\footnote%
\def\footnote{\protect\rmarkdownfootnote}

%%% Change title format to be more compact
\usepackage{titling}

% Create subtitle command for use in maketitle
\newcommand{\subtitle}[1]{
  \posttitle{
    \begin{center}\large#1\end{center}
    }
}

\setlength{\droptitle}{-2em}

  \title{House Price Precictions R Notebook}
    \pretitle{\vspace{\droptitle}\centering\huge}
  \posttitle{\par}
    \author{}
    \preauthor{}\postauthor{}
    \date{}
    \predate{}\postdate{}
  

\begin{document}
\maketitle

\section{load libraries}\label{load-libraries}

\begin{Shaded}
\begin{Highlighting}[]
\KeywordTok{library}\NormalTok{(knitr)}
\KeywordTok{library}\NormalTok{(ggplot2)}
\KeywordTok{library}\NormalTok{(plyr)}
\KeywordTok{library}\NormalTok{(dplyr)}
\end{Highlighting}
\end{Shaded}

\begin{verbatim}
## 
## Attaching package: 'dplyr'
\end{verbatim}

\begin{verbatim}
## The following objects are masked from 'package:plyr':
## 
##     arrange, count, desc, failwith, id, mutate, rename, summarise,
##     summarize
\end{verbatim}

\begin{verbatim}
## The following objects are masked from 'package:stats':
## 
##     filter, lag
\end{verbatim}

\begin{verbatim}
## The following objects are masked from 'package:base':
## 
##     intersect, setdiff, setequal, union
\end{verbatim}

\begin{Shaded}
\begin{Highlighting}[]
\KeywordTok{library}\NormalTok{(corrplot)}
\end{Highlighting}
\end{Shaded}

\begin{verbatim}
## corrplot 0.84 loaded
\end{verbatim}

\begin{Shaded}
\begin{Highlighting}[]
\KeywordTok{library}\NormalTok{(gridExtra)}
\end{Highlighting}
\end{Shaded}

\begin{verbatim}
## 
## Attaching package: 'gridExtra'
\end{verbatim}

\begin{verbatim}
## The following object is masked from 'package:dplyr':
## 
##     combine
\end{verbatim}

\begin{Shaded}
\begin{Highlighting}[]
\KeywordTok{library}\NormalTok{(scales)}
\KeywordTok{library}\NormalTok{(Rmisc)}
\end{Highlighting}
\end{Shaded}

\begin{verbatim}
## Loading required package: lattice
\end{verbatim}

\begin{Shaded}
\begin{Highlighting}[]
\KeywordTok{library}\NormalTok{(ggrepel)}
\KeywordTok{library}\NormalTok{(randomForest)}
\end{Highlighting}
\end{Shaded}

\begin{verbatim}
## randomForest 4.6-14
\end{verbatim}

\begin{verbatim}
## Type rfNews() to see new features/changes/bug fixes.
\end{verbatim}

\begin{verbatim}
## 
## Attaching package: 'randomForest'
\end{verbatim}

\begin{verbatim}
## The following object is masked from 'package:gridExtra':
## 
##     combine
\end{verbatim}

\begin{verbatim}
## The following object is masked from 'package:dplyr':
## 
##     combine
\end{verbatim}

\begin{verbatim}
## The following object is masked from 'package:ggplot2':
## 
##     margin
\end{verbatim}

\begin{Shaded}
\begin{Highlighting}[]
\KeywordTok{library}\NormalTok{(psych)}
\end{Highlighting}
\end{Shaded}

\begin{verbatim}
## 
## Attaching package: 'psych'
\end{verbatim}

\begin{verbatim}
## The following object is masked from 'package:randomForest':
## 
##     outlier
\end{verbatim}

\begin{verbatim}
## The following objects are masked from 'package:scales':
## 
##     alpha, rescale
\end{verbatim}

\begin{verbatim}
## The following objects are masked from 'package:ggplot2':
## 
##     %+%, alpha
\end{verbatim}

\begin{Shaded}
\begin{Highlighting}[]
\KeywordTok{library}\NormalTok{(xgboost)}
\end{Highlighting}
\end{Shaded}

\begin{verbatim}
## 
## Attaching package: 'xgboost'
\end{verbatim}

\begin{verbatim}
## The following object is masked from 'package:dplyr':
## 
##     slice
\end{verbatim}

\section{Read Train and Test data}\label{read-train-and-test-data}

\begin{Shaded}
\begin{Highlighting}[]
\NormalTok{train <-}\StringTok{ }\KeywordTok{read.csv}\NormalTok{(}\StringTok{"train.csv"}\NormalTok{, }\DataTypeTok{stringsAsFactors =}\NormalTok{ F)}
\NormalTok{test <-}\StringTok{ }\KeywordTok{read.csv}\NormalTok{(}\StringTok{"test.csv"}\NormalTok{, }\DataTypeTok{stringsAsFactors =}\NormalTok{ F)}
\end{Highlighting}
\end{Shaded}

\begin{Shaded}
\begin{Highlighting}[]
\KeywordTok{dim}\NormalTok{(train)}
\end{Highlighting}
\end{Shaded}

\begin{verbatim}
## [1] 1460   81
\end{verbatim}

\begin{Shaded}
\begin{Highlighting}[]
\KeywordTok{str}\NormalTok{(train[,}\KeywordTok{c}\NormalTok{(}\DecValTok{1}\OperatorTok{:}\DecValTok{10}\NormalTok{, }\DecValTok{81}\NormalTok{)])}
\end{Highlighting}
\end{Shaded}

\begin{verbatim}
## 'data.frame':    1460 obs. of  11 variables:
##  $ Id         : int  1 2 3 4 5 6 7 8 9 10 ...
##  $ MSSubClass : int  60 20 60 70 60 50 20 60 50 190 ...
##  $ MSZoning   : chr  "RL" "RL" "RL" "RL" ...
##  $ LotFrontage: int  65 80 68 60 84 85 75 NA 51 50 ...
##  $ LotArea    : int  8450 9600 11250 9550 14260 14115 10084 10382 6120 7420 ...
##  $ Street     : chr  "Pave" "Pave" "Pave" "Pave" ...
##  $ Alley      : chr  NA NA NA NA ...
##  $ LotShape   : chr  "Reg" "Reg" "IR1" "IR1" ...
##  $ LandContour: chr  "Lvl" "Lvl" "Lvl" "Lvl" ...
##  $ Utilities  : chr  "AllPub" "AllPub" "AllPub" "AllPub" ...
##  $ SalePrice  : int  208500 181500 223500 140000 250000 143000 307000 200000 129900 118000 ...
\end{verbatim}

\begin{Shaded}
\begin{Highlighting}[]
\CommentTok{# #Getting rid of the IDs but keeping the test IDs in a vector. These are needed to compose the submission file}
\NormalTok{test_labels <-}\StringTok{ }\NormalTok{test}\OperatorTok{$}\NormalTok{Id}
\NormalTok{test}\OperatorTok{$}\NormalTok{Id <-}\StringTok{ }\OtherTok{NULL}
\NormalTok{train}\OperatorTok{$}\NormalTok{Id <-}\StringTok{ }\OtherTok{NULL}
\end{Highlighting}
\end{Shaded}

\begin{Shaded}
\begin{Highlighting}[]
\NormalTok{test}\OperatorTok{$}\NormalTok{SalePrice <-}\StringTok{ }\OtherTok{NA}
\NormalTok{all <-}\StringTok{ }\KeywordTok{rbind}\NormalTok{(train, test)}
\KeywordTok{dim}\NormalTok{(all)}
\end{Highlighting}
\end{Shaded}

\begin{verbatim}
## [1] 2919   80
\end{verbatim}

\subsection{Exploring some of the most important
variables}\label{exploring-some-of-the-most-important-variables}

\begin{Shaded}
\begin{Highlighting}[]
\KeywordTok{ggplot}\NormalTok{(}\DataTypeTok{data =}\NormalTok{ all[}\OperatorTok{!}\KeywordTok{is.na}\NormalTok{(all}\OperatorTok{$}\NormalTok{SalePrice),], }\KeywordTok{aes}\NormalTok{(}\DataTypeTok{x =}\NormalTok{ SalePrice))}\OperatorTok{+}
\StringTok{  }\KeywordTok{geom_histogram}\NormalTok{(}\DataTypeTok{fill =} \StringTok{"blue"}\NormalTok{, }\DataTypeTok{binwidth =} \DecValTok{10000}\NormalTok{)}\OperatorTok{+}
\StringTok{  }\KeywordTok{scale_x_continuous}\NormalTok{(}\DataTypeTok{breaks =} \KeywordTok{seq}\NormalTok{(}\DecValTok{0}\NormalTok{, }\DecValTok{800000}\NormalTok{, }\DataTypeTok{by =} \DecValTok{100000}\NormalTok{), }\DataTypeTok{labels =}\NormalTok{ comma)}
\end{Highlighting}
\end{Shaded}

\includegraphics{House_Price_Predictions_files/figure-latex/unnamed-chunk-5-1.pdf}

\begin{Shaded}
\begin{Highlighting}[]
\KeywordTok{summary}\NormalTok{(all}\OperatorTok{$}\NormalTok{SalePrice)}
\end{Highlighting}
\end{Shaded}

\begin{verbatim}
##    Min. 1st Qu.  Median    Mean 3rd Qu.    Max.    NA's 
##   34900  129975  163000  180921  214000  755000    1459
\end{verbatim}

\subsubsection{The character variables need some work before I can use
them. To get a feel for the dataset, I decided to first see which
numeric variables have a high correlation with the
SalePrice.}\label{the-character-variables-need-some-work-before-i-can-use-them.-to-get-a-feel-for-the-dataset-i-decided-to-first-see-which-numeric-variables-have-a-high-correlation-with-the-saleprice.}

\subsubsection{Correlations with
SalePrice}\label{correlations-with-saleprice}

\begin{Shaded}
\begin{Highlighting}[]
\CommentTok{#index vector numeric variables}
\NormalTok{numericVars <-}\StringTok{ }\KeywordTok{which}\NormalTok{(}\KeywordTok{sapply}\NormalTok{(all, is.numeric))}
\CommentTok{#saving names vector for use later on}
\NormalTok{numericNames <-}\StringTok{ }\KeywordTok{names}\NormalTok{(numericVars)}
\NormalTok{## There are 37 numeric variables}
\end{Highlighting}
\end{Shaded}

\begin{Shaded}
\begin{Highlighting}[]
\NormalTok{all_numVar <-}\StringTok{ }\NormalTok{all[, numericVars]}
\CommentTok{#correlations of all numeric variables}
\NormalTok{cor_numVar <-}\StringTok{ }\KeywordTok{cor}\NormalTok{(all_numVar, }\DataTypeTok{use =} \StringTok{"pairwise.complete.obs"}\NormalTok{)}
\CommentTok{#sort on decreasing correlations with SalePrice}
\NormalTok{cor_sorted <-}\StringTok{ }\KeywordTok{as.matrix}\NormalTok{(}\KeywordTok{sort}\NormalTok{(cor_numVar[, }\StringTok{'SalePrice'}\NormalTok{], }\DataTypeTok{decreasing =} \OtherTok{TRUE}\NormalTok{))}
\CommentTok{#select only high corelations}
\NormalTok{CorHigh <-}\StringTok{ }\KeywordTok{names}\NormalTok{(}\KeywordTok{which}\NormalTok{(}\KeywordTok{apply}\NormalTok{(cor_sorted, }\DecValTok{1}\NormalTok{, }\ControlFlowTok{function}\NormalTok{(x) }\KeywordTok{abs}\NormalTok{(x) }\OperatorTok{>}\StringTok{ }\FloatTok{0.5}\NormalTok{)))}
\NormalTok{cor_numVar <-}\StringTok{ }\NormalTok{cor_numVar[CorHigh, CorHigh]}

\KeywordTok{corrplot.mixed}\NormalTok{(cor_numVar, }\DataTypeTok{tl.col =} \StringTok{"black"}\NormalTok{, }\DataTypeTok{tl.pos =} \StringTok{"lt"}\NormalTok{)}
\end{Highlighting}
\end{Shaded}

\includegraphics{House_Price_Predictions_files/figure-latex/unnamed-chunk-8-1.pdf}


\end{document}
